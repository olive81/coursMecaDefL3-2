\documentclass[%
	final, %
	 10pt, %
 	compress, % 
hyperref={bookmarks=true}	
]{beamer}

\mode<presentation>
\usetheme{Szeged}
\usecolortheme{beaver}
\usepackage{luatextra}
\setmainfont{TeX Gyre Pagella}
\usepackage{qtxmath}
\usepackage{amsmath,amssymb}
\usepackage{bm}
\usepackage{polyglossia}
\setmainlanguage{french}
\usepackage{morewrites}
\usepackage[fixamsmath]{mathtools}
\usepackage{cancel} 
\usepackage{empheq}
\usepackage{flexisym}
\usepackage{breqn}
\usepackage{numprint}
\usepackage{tabu}
\usepackage[version=3]{mhchem}
\usepackage[french=guillemets*]{csquotes} 
\MakeOuterQuote{"} 
\frenchspacing
\usepackage[backend=biber,bibencoding=utf8,natbib=true,language=french,sortlocale=fr_FR,maxitems=3,hyperref=auto,useprefix=true,style=numeric,doi=false,isbn=false,backref=false,url=false,firstinits=true,sorting=none,clearlang=true,autolang=hyphen,defernumbers=true,citecounter=true,autocite=superscript]{biblatex}
\DeclareRedundantLanguages{french}{french}
\DeclareLanguageMapping{french}{french-apa}
\usepackage{url}
\usepackage[bookmarks=true]{hyperref}
\hypersetup{  
    breaklinks={true},
    pdfpagemode=FullScreen,
    pdftoolbar={true},
    pdfmenubar=true, 
    pdffitwindow=false,   
    pdfstartview={FitH},
    pdftitle={Cours de mécanique des fluides L3A},    % title
    pdfauthor={Olivier Farges},     % author
    pdfsubject={mecadefL3A},   % subject of the document
    pdfcreator={Olivier Farges},   % creator of the document
    pdfproducer={lualatex}, % producer of the document
    pdfnewwindow=true,      % links in new window 
    colorlinks=true,       % false: boxed links; true: colored links
    linkcolor=blue,          % color of internal links (change box color with linkbordercolor)
    citecolor=green,        % color of links to bibliography
    filecolor=magenta,      % color of file links
    urlcolor=cyan           % color of external links
}
\usepackage{pgfplots}
\usepackage{textcomp}
\usepackage{tikz}
\usetikzlibrary{calc,arrows,chains,positioning,intersections,patterns,fadings,through,backgrounds,decorations.pathreplacing,3d,shapes}
\usepackage{tikz-3dplot}
\usepackage{tkz-euclide} 
\usetkzobj{all}   
\usepackage{relsize}
\usepackage{graphicx}
\usepackage{color}
\usepackage{xcolor}
\usepackage[table-align-exponent=true]{siunitx}
\usepackage{etex}
\usepackage[absolute,overlay]{textpos}
\usepackage[
     type={CC},
     modifier={by-nc-sa},
     version={3.0},
]{doclicense}
\usepackage{pythontex}
\renewcommand{\v}[1]{\ensuremath{\bm{#1}}} % for vectors
\newcommand{\ddr}[1]{\, \mathrm{d} #1} 
\newcommand{\grad}[1]{\bm{grad} \, {#1}}
\newcommand{\gradd}[1]{\overline{\overline{grad}} \, \bm{#1}}
\newcommand{\diver}[1]{\textrm{div} \ {#1}}

%%%%%%%%%%%%%%%%%%%%%%%%%%%%%%%%%%%%%%%

\newcommand\pgfmathsinandcos[3]{%
  \pgfmathsetmacro#1{sin(#3)}%
  \pgfmathsetmacro#2{cos(#3)}%
}
\newcommand\LongitudePlane[3][current plane]{%
  \pgfmathsinandcos\sinEl\cosEl{#2} % elevation
  \pgfmathsinandcos\sint\cost{#3} % azimuth
  \tikzset{#1/.style={cm={\cost,\sint*\sinEl,0,\cosEl,(0,0)}}}
}
\newcommand\LatitudePlane[3][current plane]{%
  \pgfmathsinandcos\sinEl\cosEl{#2} % elevation
  \pgfmathsinandcos\sint\cost{#3} % latitude
  \pgfmathsetmacro\yshift{\cosEl*\sint}
  \tikzset{#1/.style={cm={\cost,0,0,\cost*\sinEl,(0,\yshift)}}} %
}
\newcommand\DrawLongitudeCircle[2][1]{
  \LongitudePlane{\angEl}{#2}
  \tikzset{current plane/.prefix style={scale=#1}}
   % angle of "visibility"
  \pgfmathsetmacro\angVis{atan(sin(#2)*cos(\angEl)/sin(\angEl))} %
  \draw[current plane,thin,black] (\angVis:1) arc (\angVis:\angVis+180:1);
  \draw[current plane,thin,dashed] (\angVis-180:1) arc (\angVis-180:\angVis:1);
}%this is fake: for drawing the grid
\newcommand\DrawLongitudeCirclered[2][1]{
  \LongitudePlane{\angEl}{#2}
  \tikzset{current plane/.prefix style={scale=#1}}
   % angle of "visibility"
  \pgfmathsetmacro\angVis{atan(sin(#2)*cos(\angEl)/sin(\angEl))} %
  \draw[current plane,red,thick] (170:1) arc (170:180:1);
  %\draw[current plane,dashed] (-50:1) arc (-50:-35:1);
}%for drawing the grid
\newcommand\DLongredd[2][1]{
  \LongitudePlane{\angEl}{#2}
  \tikzset{current plane/.prefix style={scale=#1}}
   % angle of "visibility"
  \pgfmathsetmacro\angVis{atan(sin(#2)*cos(\angEl)/sin(\angEl))} %
  \draw[current plane,black,dashed, ultra thick] (150:1) arc (150:180:1);
}
\newcommand\DLatred[2][1]{
  \LatitudePlane{\angEl}{#2}
  \tikzset{current plane/.prefix style={scale=#1}}
  \pgfmathsetmacro\sinVis{sin(#2)/cos(#2)*sin(\angEl)/cos(\angEl)}
  % angle of "visibility"
  \pgfmathsetmacro\angVis{asin(min(1,max(\sinVis,-1)))}
  \draw[current plane,dashed,black,ultra thick] (-50:1) arc (-50:-35:1);

}
\newcommand\fillred[2][1]{
  \LongitudePlane{\angEl}{#2}
  \tikzset{current plane/.prefix style={scale=#1}}
   % angle of "visibility"
  \pgfmathsetmacro\angVis{atan(sin(#2)*cos(\angEl)/sin(\angEl))} %
  \draw[current plane,red,thin] (\angVis:1) arc (\angVis:\angVis+180:1);

}

\newcommand\DrawLatitudeCircle[2][1]{
  \LatitudePlane{\angEl}{#2}
  \tikzset{current plane/.prefix style={scale=#1}}
  \pgfmathsetmacro\sinVis{sin(#2)/cos(#2)*sin(\angEl)/cos(\angEl)}
  % angle of "visibility"
  \pgfmathsetmacro\angVis{asin(min(1,max(\sinVis,-1)))}
  \draw[current plane,thin,black] (\angVis:1) arc (\angVis:-\angVis-180:1);
  \draw[current plane,thin,dashed] (180-\angVis:1) arc (180-\angVis:\angVis:1);
}%Defining functions to draw limited latitude circles (for the red mesh)
\newcommand\DrawLatitudeCirclered[2][1]{
  \LatitudePlane{\angEl}{#2}
  \tikzset{current plane/.prefix style={scale=#1}}
  \pgfmathsetmacro\sinVis{sin(#2)/cos(#2)*sin(\angEl)/cos(\angEl)}
  % angle of "visibility"
  \pgfmathsetmacro\angVis{asin(min(1,max(\sinVis,-1)))}
  %\draw[current plane,red,thick] (-\angVis-50:1) arc (-\angVis-50:-\angVis-20:1);
\draw[current plane,red,thick] (-45:1) arc (-45:-35:1);

}

\tikzset{%
  >=latex,
  inner sep=0pt,%
  outer sep=2pt,%
  mark coordinate/.style={inner sep=0pt,outer sep=0pt,minimum size=3pt,
    fill=black,circle}%
}


%%%%%%%%%%%%%%%%%%%%%%%%%%%%%%%%%%%%



 \title{Mécanique des fluides}
  \author{\textsc{O.~Farges\inst{1}}}
\institute{\tiny \inst{1}	Centre RAPSODEE, UMR CNRS 5302, École des Mines d'Albi-Carmaux, 81013 Albi Cedex 09, France}
\subtitle{Cours 1 : Généralités, hydrostatique}
\date{\scriptsize \doclicenseThis}
\logo{\includegraphics[height=.75cm]{MINES.eps}}

 \AtBeginSection[]{
 \begin{frame}
  \begin{columns}[t]
   \begin{column}{5cm}
   \scriptsize{\tableofcontents[sections={2-4},currentsection,hideothersubsections]}%,hideallsubsections]}
   \end{column}
   \begin{column}{5cm}
   \scriptsize{\tableofcontents[sections={5-9},currentsection,hideothersubsections]}%,hideallsubsections]}
   \end{column}
   \end{columns}
 \end{frame}}



\begin{document}

\maketitle{}

 \section*{Sommaire}
 \begin{frame}
  \begin{columns}[t]
   \begin{column}{5cm}
   \scriptsize{\tableofcontents[sections={2-4},currentsubsection,hideallsubsections]}
   \end{column}
   \begin{column}{5cm}   \scriptsize{\tableofcontents[sections={5-9},currentsubsection,hideallsubsections]}
   \end{column}
   \end{columns}
 \end{frame}




\section{Objectifs du cours}
\label{sec:intr-au-cours}


\begin{frame}\frametitle{Objectif du cours}
Mettre en application les principes généraux de la mécanique des
  fluides :

\begin{itemize}
\item connaître les principales classes d'écoulements 
\item le modèle de fluide parfait et ses limites
\end{itemize}

\end{frame}


\section{Introduction}
\label{sec:introduction}




\subsection{Définition d'un fluide}
\label{sec:defin-dun-fluide}




\begin{frame}\frametitle{Qu'est ce qu'un fluide ?}
\begin{itemize}
\item Un milieu indéfiniment déformable : pas de forme propre
\item Se déplace et se déforme continuement tant qu'on applique une
  contrainte
\item Généralement répartis en gaz ou en liquide
\item Limite fluide-solide parfois floue :
\begin{itemize}
\item dynamique de la sollicitation (sable mouillé, ...)
\item états semi-ordonnés (cristaux liquides, ...)
\item échelle de temps (glacier)
\end{itemize}
\end{itemize}
\end{frame}

\begin{frame}\frametitle{Exemples}
\begin{itemize}
\item Monophasiques : eau, air, huile, ...
\item Multiphasiques : aérosols, émulsions, suspensions, ...
\item \enquote{Complexes} : magma, plasmas, polymères, milieux
  granulaires, ...
\end{itemize}
\end{frame}



\subsection{Domaines d'application}
\label{sec:doma-dappl}

\begin{frame}\frametitle{Pourquoi étudier les fluides ?}
\begin{itemize}
\item \'{E}{}nergétique : fluides de transfert, combustion
\item Transport de fluides : eau, gaz naturel
\item Météorologie : étude des masses d'air, d'eau
\item Biologie : sang,...
\item Aéronautique : aérodynamique, motorisations, ...
\item Agroalimentaire
\item Pharmacie : réacteurs, ...
\item et tout autre domaine du génie des procédés qui est confronté à un fluide !
\end{itemize}
\end{frame}


\subsection{Caractéristiques d'un fluide}
\label{sec:caracteristiques}


\begin{frame}\frametitle{Différentes échelles d'observation}
\begin{itemize}
\item Micro :
\begin{itemize}
\item Atomes et molécules en interaction 
\end{itemize}
\item Méso : notion de particule de fluide
\begin{itemize}
\item Grand : beaucoup de molécules
\item Petit : homogénéïté spatiale des propriétés physiques
\end{itemize}
\item Macro
\begin{itemize}
\item Milieu continu
\item subit/exerce des forces par/sur son environnement 
\end{itemize}
\end{itemize}
\end{frame}

\begin{frame}\frametitle{Différentes échelles d'observation}
\begin{figure}
	\begin{tikzpicture}[scale=0.75,every node/.style={minimum size=1cm}]
	%% some definitions
	\tkzDefPoints{3.75/-1.5/a1,
              4/1/a2,
              1.25/-0.25/a3,
              -1/-3/a4}




	\def\R{4} % sphere radius

	
	\def\angEl{25} % elevation angle
	\def\angAz{-100} % azimuth angle
	\def\angPhiOne{-45} % longitude of point P
	\def\angPhiTwo{-35} % longitude of point Q
	\def\angBeta{10} % latitude of point P and Q
	
	%% working planes
	
	\pgfmathsetmacro\H{\R*cos(\angEl)} % distance to north pole
	\LongitudePlane[xzplane]{\angEl}{\angAz}
	\LongitudePlane[pzplane]{\angEl}{\angPhiOne}
	\LongitudePlane[qzplane]{\angEl}{\angPhiTwo}
	\LatitudePlane[equator]{\angEl}{0}
	\fill[ball color=white!10] (0,0) circle (\R); % 3D lighting effect
	\coordinate (O) at (0,0);
	\coordinate[mark coordinate] (N) at (0,\H);
	\coordinate[mark coordinate] (S) at (0,-\H);
	\path[xzplane] (\R,0) coordinate (XE);
	
    %defining points outsided the area bounded by the sphere
	\path[qzplane] (\angBeta:\R+5.2376) coordinate (XEd);
	\path[pzplane] (\angBeta:\R) coordinate (P);%fino alla sfera
	\path[pzplane] (\angBeta:\R+5.2376) coordinate (Pd);%sfora di una quantit pari a 10 dopo la sfera
	\path[pzplane] (\angBeta:\R+5.2376) coordinate (Td);%sfora di una quantit pari a 10 dopo la sfera
	\path[pzplane] (\R,0) coordinate (PE);
    \path[pzplane] (\R+4,0) coordinate (PEd);
	\path[qzplane] (\angBeta:\R) coordinate (Q);
	\path[qzplane] (\angBeta:\R) coordinate (Qd);%sfora di una quantit pari a 10 dopo la sfera
	
	\path[qzplane] (\R,0) coordinate (QE);
    \path[qzplane] (\R+4,0) coordinate (QEd);%sfora di una quantit 10 dalla sfera sul piano equat.


    \DrawLongitudeCircle[\R]{\angPhiOne} % pzplane
    \DrawLongitudeCircle[\R]{\angPhiTwo} % qzplane
    \DrawLatitudeCircle[\R]{\angBeta}
    \DrawLatitudeCircle[\R]{0} % equator
	
	\draw[-,dashed, thick] (N) -- (S);

    %     		\path[xzplane] (0:\R) node[below] {$$};
    %     \path[xzplane] (\angBeta:\R) node[below left] {$$};
    % \foreach \t in {0,2,...,10} { \DrawLatitudeCirclered[\R]{\t} }
    %     \foreach \t in {135,137,...,145} { \DrawLongitudeCirclered[\R]{\t} }
	
\begin{scope}[shift={(-2,-2,-2)}]
    \draw[right color=gray!10, left color=black!80] (2.5,2.5,3) -- (2.5,3,3) -- (3,3,3) -- (3,2.5,3) -- cycle;
   \draw[bottom color=gray!10, top color=black!80] (2.5,3,3) --
   (2.5,3,2.5) --   (3,3,2.5) -- (3,3,3) -- cycle ;
   \draw[left color=gray!10, right color=black!80] (3,2.5,2.5) --
   (3,2.5,3)  -- (3,3,3)-- (3,3,2.5) -- cycle;
\end{scope}
%\tkzLabelPoint[above = 2pt  of a1, color = blue](a1){$dS$}
%\tkzLabelPoint[above = 2pt  of a2, color = blue](a2){$S$}
\tkzLabelPoint[above = 2pt  of a3, color = red](a3){$dV$}
\tkzLabelPoint[above = 2pt  of a4, color = red](a4){$V$}
    	
\end{tikzpicture}
\end{figure}
\end{frame}


\begin{frame}\frametitle{Milieu continu }
\begin{itemize}
\item Grandeur extensive $G(t)$ dans le volume $V$ $\rightarrow$ \textcolor{red}{GLOBALE}
  (\textit{augmente avec le nombre de molécules})
\item Grandeur volumique $g(x,y,z,t) = \dfrac{\ddr G}{\ddr V}$ $\ddr
  V$ $\rightarrow$ $G(t) = \iiint\limits_{V} g \ddr V  $ $\rightarrow$
  \textcolor{blue}{LOCALE}
\item Exemples :
\begin{itemize}
\item Masse de fluide $M$ dans $V$ : ${\color{red} M(t)} = \iiint\limits_{V} \textcolor{blue}{\rho}
  \ddr V$
 \item Quantité de mouvement $\v{P}$ dans $V$ : ${\color{red}\v{P}(t)} = \iiint\limits_{V}
   \textcolor{blue}{\rho \v{v}}   \ddr V$
 \item Énergie cinétique $K$ dans $V$ : ${\color{red}K(t)} = \iiint\limits_{V}
  \dfrac{1}{2}\textcolor{blue}{\rho v}^{2} \ddr V$
 \item Énergie interne $U$ dans $V$ : ${\color{red}U(t)} = \iiint\limits_{V}
   \textcolor{blue}{\rho u} \ddr V$
 \end{itemize}
\end{itemize}
\end{frame}


\begin{frame}\frametitle{La masse volumique}
\begin{itemize}
\item à priori non uniforme dans l'espace :
\begin{itemize}
\item Varie avec la temprérature $\Rightarrow$ dilatabilité
\item Varie avec la pression $\Rightarrow$ compressibilité
\end{itemize}
\item Approximation bien utile : le fluide incompressible
\end{itemize}
\end{frame}


\begin{frame}\frametitle{Descriptions lagrangienne et eulérienne}
\begin{description}
\item[Lagrangienne] suivi des particules en fonction
  de l'espace et du temps $\Rightarrow$ nécessite la connaissance de
  la configuration initiale
\item[Eulérienne] variations dans le temps des caractéristiques de
  l'écoulement en des points fixes de l'espace   $\Rightarrow$ donne
  accès à la répartition des vitesses et des températures dans le fluide
\end{description}
\end{frame}

\begin{frame}\frametitle{Dérivée particulaire}
\begin{itemize}
\item scalaire : 
\begin{dmath*}
\dfrac{\ddr b}{ \ddr t} = \dfrac{\partial
    b}{\partial t} + \v{v} \cdot \grad{b}
\end{dmath*}
\item fonction vectorielle : 
\begin{dmath*}
\dfrac{\ddr \v{b}}{ \ddr t} = \dfrac{\partial
    \v{b}}{\partial t} + \left(\gradd{b} \cdot \v{v}\right) 
\end{dmath*}
\item intégrale triple : 
\begin{dmath*}
\dfrac{\ddr}{ \ddr t} \iiint_{V} f \ddr V \hiderel{=}
  \iiint_{V} \left( \dfrac{\ddr f}{\ddr t} \hiderel{+} f \diver{\v{v}} \right)
 \ddr V
\end{dmath*}
\begin{dmath*}
\dfrac{\ddr}{ \ddr t} \iiint_{V} f \ddr V \hiderel{=} \iiint_{V} \left(  \dfrac{\partial f}{\partial t} \hiderel{+} \diver{( f \v{v})} \right)
 \ddr V
\end{dmath*}
\end{itemize}
\end{frame}

\subsection{Quelques images}
\label{sec:quelques-images}


\begin{frame}\frametitle{}


\begin{tikzpicture}
\node[anchor=south west,inner sep=0pt] at (5,2) {\includegraphics[height=3cm]{images/Catarina.jpg}};
\node[anchor=south west,inner sep=0pt] at (-1,2) {\includegraphics[height=3cm]{images/avion.jpg}};
\node[anchor=south west,inner sep=0pt] at (5,6) {\includegraphics[height=3cm]{images/barnard.jpg}};
\node[anchor=south west,inner sep=0pt] at (-1,6) {\includegraphics[height=3cm]{images/vonkarman.jpg}};
\end{tikzpicture}
\end{frame}



\section{Cinématique des fluides}
\label{sec:cinem-des-fluid}

\begin{frame}\frametitle{Lignes de courant et trajectoires}
\begin{description}
\item [Lignes de courant :] Lignes du champ de vecteurs $\v{v}(\v{r},
  t)$ tangentes en chaque point au vecteur vitesse
  $\v{v}(x,y,z,t_{0})$ à un instant $t_{0}$
\item[Trajectoire :] le chemin suivi par une particule au cours du temps 
\end{description}
La ligne de courant est relative à \emph{un même instant} et regroupe
\emph{plusieurs particules}, la trajectoire se réfère à \emph{une seule particule}
sur \emph{une période de temps}
\end{frame}

\begin{frame}\frametitle{Nombre de Reynolds}
\begin{itemize}
\item Nombre adimensionnel
\item Rapport entre les effets d'inertie et les effets visqueux
\item Permet de définir les effets dominants $\Rightarrow$ possibilité
  de simplification
\item Définit la limite entre écoulements laminaire et turbulent
\end{itemize}
\begin{dmath*}
Re = \dfrac{\rho L v}{\mu} 
\end{dmath*}
\begin{dmath*}
Re = \dfrac{\textcolor{red}{\rho v^{2} / L}}{\textcolor{blue}{\mu v /L}}
\end{dmath*}
\textcolor{red}{Transport convectif de quantité de mouvement}

\textcolor{blue}{Transport diffusif de quantité de mouvement}
\end{frame}


\begin{frame}\frametitle{Ecoulements}
\begin{itemize}
\item $Re < 2000$ $\Rightarrow$ écoulement laminaire : vitesse
  régulière dans le temps et l'espace
\item $Re > 3000$ $\Rightarrow$ écoulement turbulent : vitesse
  chaotique dans l'espace et le temps
\item écoulement stationnaire ou permanent : dérivée partielle par
  rapport au temps nulle $\Rightarrow$ dérivée particulaire limitée à
  sa partie convective $\v{v} \cdot \grad{ }$
\item écoulements bidimensionnels ou de révolution, écoulements irrotationnels
\end{itemize}
\end{frame}


\begin{frame}\frametitle{Débits $Q_{m}$ et $Q_{v}$}
\begin{itemize}
\item Débit massique : masse de matière traversant toute surface
  pendant l'unité de temps 
\begin{dmath*}
Q_{m} = \iint_{S} \rho \v{v} \cdot \v{n} \ddr S
\end{dmath*}
\item Débit volumique : volume de matière  toute surface
  pendant l'unité de temps 
\begin{dmath*}
Q_{v} = \dfrac{Q_{m}}{\rho} \hiderel{=} \iint_{S} \v{v} \cdot \v{n} \ddr S
\end{dmath*}
\item En régime stationnaire le débit massique se conserve
\end{itemize}
\end{frame}




\section{Fluide et forces}
\label{sec:fluide-et-forces}

\begin{frame}\frametitle{Deux types de forces}
\begin{figure}
	\begin{tikzpicture}[scale=0.35,every node/.style={minimum size=1cm}]
	%% some definitions
	\tkzDefPoints{3.75/-1.5/a1,
              4/1/a2,
              1.25/-0.25/a3,
              -1/-3/a4}
	\def\R{4} % sphere radius
	\def\angEl{25} % elevation angle
	\def\angAz{-100} % azimuth angle
	\def\angPhiOne{-45} % longitude of point P
	\def\angPhiTwo{-35} % longitude of point Q
	\def\angBeta{10} % latitude of point P and Q
	\pgfmathsetmacro\H{\R*cos(\angEl)} % distance to north pole
	\LongitudePlane[xzplane]{\angEl}{\angAz}
	\LongitudePlane[pzplane]{\angEl}{\angPhiOne}
	\LongitudePlane[qzplane]{\angEl}{\angPhiTwo}
	\LatitudePlane[equator]{\angEl}{0}
	\fill[ball color=white!10] (0,0) circle (\R); % 3D lighting effect
	\coordinate (O) at (0,0);
	\coordinate[mark coordinate] (N) at (0,\H);
	\coordinate[mark coordinate] (S) at (0,-\H);
	\path[xzplane] (\R,0) coordinate (XE);
	\path[qzplane] (\angBeta:\R+5.2376) coordinate (XEd);
	\path[pzplane] (\angBeta:\R) coordinate (P);%fino alla sfera
	\path[pzplane] (\angBeta:\R+5.2376) coordinate (Pd);%sfora di una quantit pari a 10 dopo la sfera
	\path[pzplane] (\angBeta:\R+5.2376) coordinate (Td);%sfora di una quantit pari a 10 dopo la sfera
	\path[pzplane] (\R,0) coordinate (PE);
    \path[pzplane] (\R+4,0) coordinate (PEd);
	\path[qzplane] (\angBeta:\R) coordinate (Q);
	\path[qzplane] (\angBeta:\R) coordinate (Qd);%sfora di una quantit pari a 10 dopo la sfera
	\path[qzplane] (\R,0) coordinate (QE);
    \path[qzplane] (\R+4,0) coordinate (QEd);%sfora di una quantit 10 dalla sfera sul piano equat.
    \DrawLongitudeCircle[\R]{\angPhiOne} % pzplane
    \DrawLongitudeCircle[\R]{\angPhiTwo} % qzplane
    \DrawLatitudeCircle[\R]{\angBeta}
    \DrawLatitudeCircle[\R]{0} % equator
	\draw[-,dashed, thick] (N) -- (S);
			\path[xzplane] (0:\R) node[below] {$$};
	\path[xzplane] (\angBeta:\R) node[below left] {$$};
    \foreach \t in {0,2,...,10} { \DrawLatitudeCirclered[\R]{\t} }
	\foreach \t in {135,137,...,145} { \DrawLongitudeCirclered[\R]{\t} }
\begin{scope}[shift={(-2,-2,-2)}]
    \draw[right color=gray!10, left color=black!80] (2.5,2.5,3) -- (2.5,3,3) -- (3,3,3) -- (3,2.5,3) -- cycle;
   \draw[bottom color=gray!10, top color=black!80] (2.5,3,3) --
   (2.5,3,2.5) --   (3,3,2.5) -- (3,3,3) -- cycle ;
   \draw[left color=gray!10, right color=black!80] (3,2.5,2.5) --
   (3,2.5,3)  -- (3,3,3)-- (3,3,2.5) -- cycle;
\end{scope}
\tkzLabelPoint[above = 2pt  of a1, color = blue](a1){$dS$}
\tkzLabelPoint[above = 2pt  of a2, color = blue](a2){$S$}
\tkzLabelPoint[above = 2pt  of a3, color = red](a3){$dV$}
\tkzLabelPoint[above = 2pt  of a4, color = red](a4){$V$}
\end{tikzpicture}
\end{figure}
\begin{description}
\item [Les forces volumiques] s'exercent sur chaque particule fluide
  interne au volume $V$
\item[Les forces surfaciques] s'exercent sur la
  frontière $S$ du volume $V$
\end{description}
\end{frame}


\subsection{Forces volumiques}
\label{sec:forces-volumiques}

\begin{frame}\frametitle{Efforts à distance}
\begin{itemize}
\item représentées par une densité massique de forces $\v{f}$
\begin{dmath*}
\v{F}_{V} = \iiint_{V} \rho \v{f} \ddr V
\end{dmath*}
\item La plus courante : la Force d'attraction gravitationnelle (le
  poids)
\item somme des poids élémentaires $\ddr m \v{g} = \rho \v{g} \ddr V $ de toutes les particules
  fluides $\ddr V$
\begin{dmath*}
\v{P} = \iiint_{V} \rho \v{g} \ddr V
\end{dmath*}
\item Forces d'inertie (référentiel non galiléen)
\item somme des forces d'inertie élémentaires $- \ddr m (\v{a}_{e} +
  \v{a}_{c} ) = - \rho \ddr V  (\v{a}_{e} +
  \v{a}_{c} )$ de toutes les particules fluides $\ddr V$
\begin{dmath*}
\v{F}_{ie} + \v{F}_{ic} = \iiint_{V} - \rho (\v{a}_{e} +
  \v{a}_{c} )  \ddr V
\end{dmath*}
\item Forces électriques, magnétiques, ...
\end{itemize}
\end{frame}

\subsection{Forces surfaciques}
\label{sec:forces-surfaciques}

\begin{frame}\frametitle{Forces de contact}
\begin{itemize}
\item Fluide au repos $\Rightarrow$ force normale à $\ddr S$ 
\begin{dmath*}
\v{F}_{p} = \iint_{S} -p  \v{n} \ddr S 
\end{dmath*}
\item pression, frottements visceux
\end{itemize}
\end{frame}

\begin{frame}\frametitle{La pression}
\begin{itemize}
\item Pour un gaz elle est liée aux chocs entre molécules : échange de quantité de mouvement
\item Pour un liquide également mais il y a des forces attractives et
  des forces répulsives
\begin{itemize}
\item un liquide a une cohésion
\item on peut \enquote{tirer dessus} sans le déchirer $\Rightarrow$
  pressions négatives
\item il peut s'accrocher et s'étaler à des solides $\Rightarrow$
  capillarité, mouillage
\end{itemize}
\end{itemize}
\end{frame}




\subsection{Hydrostatique}
\label{sec:hydrostatique-1}



\begin{frame}\frametitle{Fluide immobile}
\begin{itemize}
\item Équilibre entre les forces de pression et les forces volumiques
\end{itemize}
\begin{alertblock}{A RETENIR} 
\begin{dmath*}
\textcolor{red}{\v{F}_{P} + \v{P} = 0}
\end{dmath*}
\begin{dmath*}
\textcolor{red}{\iint_{S} - p  \v{n} \ddr S +  \iiint_{V} \rho \v{g} \ddr V = 0}
\end{dmath*}
\end{alertblock}
\end{frame}


\begin{frame}\frametitle{Conséquences}
\begin{dmath*}
\grad{p} = \rho \v{g}
\end{dmath*}
\begin{itemize}
\item Peut être intégrer $\Rightarrow$ champ de pression
  $\v{p}(x,y,z)$ dans un fluide au repos
\item CL : $p = p_{atm}$ sur la surface de contact avec l'air
\item Les surfaces isobares sont perpendiculaires à $\v{g}$ 
\item La pression augmente le long de $\v{g}$ (plongeur)
\item En toute logique, elle diminue en sens inverse (cabines d'avion)
\end{itemize}
% \begin{alertblock}{A RETENIR} 
% \begin{dmath*}
% \textcolor{red}{\v{F}_{P} + \v{P} = 0
% \iint_{S} - p  \v{n} \ddr S +  \iiint_{V} \rho \v{g} \ddr V = 0}
% \end{dmath*}
% \end{alertblock}
\end{frame}




% \begin{frame}\frametitle{}
% \begin{center}
% \begin{tikzpicture}
% \tkzDefPoints{1.75/3.5/a1,
%               4/2/a2,
%               1.5/1.25/a3,
%               3.5/2.5/a4}
% \draw [thick,fill=gray!40] (2,4.5) ..controls +(-1,0) and
% +(-1,1.5).. (1,1.5) ..controls +(0.7,-1) and +(0,-0.5).. (3,1)
% ..controls +(0,2) and +(1,-1).. (4,3.5) ..controls +(-1,1) and
% +(0.5,0).. (2,4.5);
% \shade[yslant=-0.25, xslant= 0.5,top color=gray!10, bottom color=black!50]
%     (0,4) rectangle +(0.5,0.5);
% \begin{scope}[shift={(0.5,0,0)}]
%     \draw[right color=gray!10, left color=black!80] (2.5,2.5,3) -- (2.5,3,3) -- (3,3,3) -- (3,2.5,3) -- cycle;
%    \draw[bottom color=gray!10, top color=black!80] (2.5,3,3) --
%    (2.5,3,2.5) --   (3,3,2.5) -- (3,3,3) -- cycle ;
%    \draw[left color=gray!10, right color=black!80] (3,2.5,2.5) --
%    (3,2.5,3)  -- (3,3,3)-- (3,3,2.5) -- cycle;
% \end{scope}
% \tkzLabelPoint[above = 2pt  of a1](a1){\small$dS$}
% \tkzLabelPoint[above = 2pt  of a2](a2){\small$S$}
% \tkzLabelPoint[above = 2pt  of a3](a3){\small$dV$}
% \tkzLabelPoint[above = 2pt  of a4](a4){\small$V$}
% \end{tikzpicture}
% \end{center}
% \end{frame}






%\section{Hydrostatique}
%\label{sec:hydrostatique}


% \begin{frame}\frametitle{}
% \begin{figure}
% 	\begin{tikzpicture}[scale=0.5,every node/.style={minimum size=1cm}]
% 	%% some definitions
% 	\tkzDefPoints{3.75/-1.5/a1,
%               4/1/a2,
%               1.25/-0.25/a3,
%               -1/-3/a4}




% 	\def\R{4} % sphere radius

	
% 	\def\angEl{25} % elevation angle
% 	\def\angAz{-100} % azimuth angle
% 	\def\angPhiOne{-45} % longitude of point P
% 	\def\angPhiTwo{-35} % longitude of point Q
% 	\def\angBeta{10} % latitude of point P and Q
	
% 	%% working planes
	
% 	\pgfmathsetmacro\H{\R*cos(\angEl)} % distance to north pole
% 	\LongitudePlane[xzplane]{\angEl}{\angAz}
% 	\LongitudePlane[pzplane]{\angEl}{\angPhiOne}
% 	\LongitudePlane[qzplane]{\angEl}{\angPhiTwo}
% 	\LatitudePlane[equator]{\angEl}{0}
% 	\fill[ball color=white!10] (0,0) circle (\R); % 3D lighting effect
% 	\coordinate (O) at (0,0);
% 	\coordinate[mark coordinate] (N) at (0,\H);
% 	\coordinate[mark coordinate] (S) at (0,-\H);
% 	\path[xzplane] (\R,0) coordinate (XE);
	
%     %defining points outsided the area bounded by the sphere
% 	\path[qzplane] (\angBeta:\R+5.2376) coordinate (XEd);
% 	\path[pzplane] (\angBeta:\R) coordinate (P);%fino alla sfera
% 	\path[pzplane] (\angBeta:\R+5.2376) coordinate (Pd);%sfora di una quantit pari a 10 dopo la sfera
% 	\path[pzplane] (\angBeta:\R+5.2376) coordinate (Td);%sfora di una quantit pari a 10 dopo la sfera
% 	\path[pzplane] (\R,0) coordinate (PE);
%     \path[pzplane] (\R+4,0) coordinate (PEd);
% 	\path[qzplane] (\angBeta:\R) coordinate (Q);
% 	\path[qzplane] (\angBeta:\R) coordinate (Qd);%sfora di una quantit pari a 10 dopo la sfera
	
% 	\path[qzplane] (\R,0) coordinate (QE);
%     \path[qzplane] (\R+4,0) coordinate (QEd);%sfora di una quantit 10 dalla sfera sul piano equat.


%     \DrawLongitudeCircle[\R]{\angPhiOne} % pzplane
%     \DrawLongitudeCircle[\R]{\angPhiTwo} % qzplane
%     \DrawLatitudeCircle[\R]{\angBeta}
%     \DrawLatitudeCircle[\R]{0} % equator
	
% 	\draw[-,dashed, thick] (N) -- (S);

% 			\path[xzplane] (0:\R) node[below] {$$};
% 	\path[xzplane] (\angBeta:\R) node[below left] {$$};
%     \foreach \t in {0,2,...,10} { \DrawLatitudeCirclered[\R]{\t} }
% 	\foreach \t in {135,137,...,145} { \DrawLongitudeCirclered[\R]{\t} }
	
% \begin{scope}[shift={(-2,-2,-2)}]
%     \draw[right color=gray!10, left color=black!80] (2.5,2.5,3) -- (2.5,3,3) -- (3,3,3) -- (3,2.5,3) -- cycle;
%    \draw[bottom color=gray!10, top color=black!80] (2.5,3,3) --
%    (2.5,3,2.5) --   (3,3,2.5) -- (3,3,3) -- cycle ;
%    \draw[left color=gray!10, right color=black!80] (3,2.5,2.5) --
%    (3,2.5,3)  -- (3,3,3)-- (3,3,2.5) -- cycle;
% \end{scope}
% \tkzLabelPoint[above = 2pt  of a1, color = blue](a1){$dS$}
% \tkzLabelPoint[above = 2pt  of a2, color = blue](a2){$S$}
% \tkzLabelPoint[above = 2pt  of a3, color = red](a3){$dV$}
% \tkzLabelPoint[above = 2pt  of a4, color = red](a4){$V$}
    	
% \end{tikzpicture}
% \end{figure}
% \begin{itemize}
% \item 
% \end{itemize}
% \end{frame}



\section{Équations bilan}
\label{sec:nombr-adim}


\begin{frame}\frametitle{3 équations pour un fluide en mouvement}
\begin{itemize}
\item Conservation de la masse
\item Conservation de la quantité de mouvement
\item Conservation de l'énergie
\end{itemize}
\end{frame}

\subsection{Conservation de la masse}
\label{sec:conservation-de-la}





\begin{frame}\frametitle{Conservation de la masse}
\begin{dmath*}
\dfrac{\partial \rho}{\partial t} + \diver{(\rho \v{v})} = 0
\end{dmath*}
\begin{itemize}
\item Si l'écoulement des incompressible, $\rho = Cte$ dans le temps
  et l'espace et l'on a simplement $\diver{(\v{v})} = 0$
\end{itemize}
\end{frame}


\subsection{Conservation de la quantité de mouvement}
\label{sec:conservation-de-la-1}

% \begin{frame}\frametitle{Conservation de la quantité de mouvement}
% \begin{dmath*}
% \dfrac{\ddr}{\ddr t} 
% \Biggl(
%  \iiint _{V} \rho \v{v} \ddr V  \Biggr) =   \underbrace{\iiint_{V} \rho \v{f} \ddr
%   V}_{\textcolor{blue}{\text{\scriptsize Forces Volumiques}}} +
%  \underbrace{\iint_{S} \overline{\overline{\sigma}} \cdot \v{n} \ddr
% S}_{\textcolor{blue}{\text{\scriptsize Forces Surfaciques}}}
% \end{dmath*}
% \end{frame}

\begin{frame}\frametitle{Conservation de la quantité de mouvement}
\begin{dmath*}
\dfrac{\ddr}{\ddr t} 
\Biggl(
 \iiint _{V} \rho \v{v} \ddr V  \Biggr) =   \underbrace{\iiint_{V} \rho \v{f} \ddr
  V}_{\textcolor{blue}{\text{\scriptsize Forces Volumiques}}} +
 \underbrace{\iint_{S} \overline{\overline{\sigma}} \cdot \v{n} \ddr
S}_{\textcolor{blue}{\text{\scriptsize Forces Surfaciques}}}
\end{dmath*}
\begin{dmath*}
\rho \dfrac{\ddr  \v{v}}{\ddr t} = \rho \v{f} +
\diver{\overline{\overline{\sigma}}}
\end{dmath*}
\begin{dmath*}
\rho \dfrac{\partial  \v{v}}{\partial t} + \rho \left( \gradd{\v{v}}
\right) \v{v} = \rho \v{f} - \grad{p} +
\diver{\overline{\overline{\tau}}}
\end{dmath*}
\small
\begin{columns}
\begin{column}{5cm}
\begin{itemize}
\item $\rho \dfrac{\partial  \v{v}}{\partial t}$ : terme
  instantionnaire
\item $\rho \left( \gradd{\v{v}}
\right) \v{v}$ : terme d'advection
\end{itemize}
\end{column}
\begin{column}{5cm}
\begin{itemize}
\item $\rho \v{f}$ : force volumique (poids)
\item $- \grad{p}$ : force surfacique (pression)
\item $\diver{\overline{\overline{\tau}}}$ : force surfacique (viscosité)
\end{itemize}
\end{column}
\end{columns}
\end{frame}



\subsection{Conservation de l'énergie}
\label{sec:cons-de-lenerg}



\begin{frame}\frametitle{Conservation de l'énergie}
\begin{itemize}
\item En toute généralité
\end{itemize}
\begin{dmath*}
\rho \dfrac{\ddr  u}{\ddr t} = -p \diver{\v{v}} +
\overline{\overline{\tau}} \hiderel{:} \gradd {v} - \diver{\v{q}} + P_{d} 
\end{dmath*}
{\small
\begin{columns}
\begin{column}{5cm}
\begin{itemize}
\item $\rho \dfrac{\ddr  u}{\ddr t}$ : taux de variation de l'énergie interne
\item $p \diver{\v{v}}$ : puissance des forces de compression
\end{itemize}
\end{column}
\begin{column}{5cm}
\begin{itemize}
\item $\overline{\overline{\tau}} : \gradd {v}$ : puissance des forces
  visqueuses ($:  \Rightarrow$ double produit scalaire)
\item $\diver{\v{q}}$ : flux de chaleur 
\item $P_{d}$ : puissance dissipée
\end{itemize}
\end{column}
\end{columns}}
\begin{itemize}
\item En ce qui nous concerne : écoulements incompressibles à faible vitesse
\end{itemize}
\begin{dmath*}
\rho \dfrac{\ddr  u}{\ddr t} =  - \diver{\v{q}} + P_{d} 
\end{dmath*}
\end{frame}


\section{Applications}
\label{sec:applications}


\subsection{La poussée d'Archimède}
\label{sec:la-pouss-darch}


\begin{frame}\frametitle{Tout corps plongé dans l'eau ...}
\begin{alertblock}{énoncé}
\enquote{\textit{Tout corps plongé dans un fluide au repos,
    entièrement mouillé par celui-ci ou traversant sa surface libre,
    subit une force verticale, dirigée de bas en haut et opposée au
    poids du volume de fluide déplacé ; cette force est appelée
    poussée d'Archimède. }}
\end{alertblock}
\end{frame}

\begin{frame}\frametitle{Tout corps plongé dans l'eau ...}
\begin{itemize}
\item C'est simplement la résultante des forces de pression
\item On cherche à connaître la force exercée par le fluide sur un
  corps étranger
\end{itemize}
\end{frame}

\begin{frame}\frametitle{Un corps dans un fluide statique}
\begin{itemize}
\item Le corps n'est pas en équilibre $\Rightarrow \v{P}_{corps} +
  \v{F}_{p} \neq 0$
\item 2 cas possibles
\begin{itemize}
\item Le corps monte : $\rho_{corps} < \rho_{fluide}$
\item Le corps descend : $\rho_{corps} > \rho_{fluide}$
\end{itemize}
\end{itemize}
\end{frame}

\begin{frame}\frametitle{Moment des forces de pression}
\begin{itemize}
\item Utile pour les problèmes de stabilité, de rotation
\item Le moment total en un point $A$ de la force de pression
  $F_{p}$ est la somme des moments élémentaires
\end{itemize}
\begin{dmath*}
M_{A} (F_{p}) = \iint_{S} AM \wedge \ddr F_{p}
\end{dmath*}
\begin{dmath*}
M_{A} (F_{p}) = \iint_{S} AM \wedge \left( -p \v{n} \ddr S \right)
\end{dmath*}
\end{frame}

\begin{frame}\frametitle{Centre de poussée}
\begin{itemize}
\item Appelé aussi centre de carène
\item C'est le point $C$ tel que $M_{C} (\v{F}_{p}) = 0$
\item C'est le centre de gravité du volume immergé
\end{itemize}
\end{frame}



\section{Références}
\label{sec:references}



\begin{frame}\frametitle{}
\begin{itemize}
\item O. Louisnard, Cours de Mécanique des fluides, 25/09/2012
  \textit{http://perso.mines-albi.fr/~louisnar/ENSEIGNEMENT/MECADEF/PolyMecaDef.pdf}
\item R. Ansart, Mécanique des fluides et transferts, Cours de 1er
  année formation apprentissage
\item F. Veynandt, Mécanique des fluides
\item M. Rl Hafi, Cours de Mécanique des fluides
\end{itemize}
\end{frame}







% \begin{frame}[fragile]
% \frametitle{Exemple de code 1}
% \scriptsize
% \begin{pyconsole}
% from sympy import *
% var("x, y")
% z = (x + y)**3
% z
% expand(z)
% latex(expand(z))
% \end{pyconsole}
% \end{frame}


% \begin{frame}[fragile]
% \frametitle{Exemple de code 2}
% \tiny
% \begin{pyconsole}
% from pylab import * 
% indepVars = arange(0, 5, 0.1) 
% depVars = sin(indepVars) 
% depVars_Error = 0.1*abs(randn(len(depVars)))
% Fit_Coeffs = polyfit(indepVars, depVars, 2) 
% depVars_Best_Fit = polyval(Fit_Coeffs, indepVars)
% figure()
% errorbar(indepVars, depVars, yerr=depVars_Error, fmt='o', label='experimental run 1') 
% plot(indepVars, depVars_Best_Fit, label='best fit quadratic polynomial') 
% xlabel('Independent Variable axis label (IV units)')
% ylabel('Dependent Variable axis label (DV units)')
% legend() 
% title('Scatter plot with error bars')  
% plt.savefig("fig2.png")
% \end{pyconsole}
% \end{frame}


\end{document}

%%% Local Variables:
%%% coding: utf-8
%%% mode: latex
%%% TeX-engine: luatex
%%% TeX-master: t
%%% End:
